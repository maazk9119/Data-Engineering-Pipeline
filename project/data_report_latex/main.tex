\documentclass[a4paper,11pt]{article}
\usepackage{graphicx}
\usepackage{amsmath}
\usepackage{hyperref}
\usepackage{geometry}
\usepackage{natbib}
\usepackage{project}
\usepackage{tikz}
\usetikzlibrary{positioning, shapes.geometric, arrows}

\begin{document}

\section{Questions}
\begin{itemize}
    \item How does neighborhood poverty level impact school enrollment rates across different regions in the United States?
    
    \item How do poverty levels around schools vary by region in the United States, and what might these differences suggest about educational inequality?
\end{itemize}

\section{Data Sources}
    \subsection{Description of Data Sources}
    \begin{itemize}
    \item \textbf{Dataset 1: School Neighborhood Poverty Estimates 2020-2021}
    
    This dataset from the National Center for Education Statistics \(\text(NCES)\) includes data on neighborhood poverty levels surrounding schools, which can serve as a proxy for educational access and community socioeconomic conditions. \cite{dataset1} 

    \item \textbf{Dataset 2: Report Card Enrollment}

    This dataset provides detailed school enrollment statistics disaggregated by school, district, and state for the 2022-23 school year. It includes student counts by demographics, which can help analyze how regional poverty levels might influence enrollment and highlight disparities across regions.\cite{dataset2}
\end{itemize}

\end{document}